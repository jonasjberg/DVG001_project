% ______________________________________________________________________________
%
% DVG001 -- Introduktion till Linux och små nätverk
%                                     Projektarbete
% ~~~~~~~~~~~~~~~~~~~~~~~~~~~~~~~~~~~~~~~~~~~~~~~~~
% Author:   Jonas Sjöberg
%           tel12jsg@student.hig.se
%
% Date:     2016-06-07 -- 2016-06-13
%
% License:  Creative Commons Attribution 4.0 International (CC BY 4.0)
%           <http://creativecommons.org/licenses/by/4.0/legalcode>
%           See LICENSE.md for additional licensing information.
% ______________________________________________________________________________


\section{Resultat}

% ~~~~~~~~~~~~~~~~~~~~~~~~~~~~~~~~~~~~~~~~~~~~~~~~~~~~~~~~~~~~~~~~~~~~~~~~~~~~~~
\section{Diskussion}
Skulle jag ha vetat det jag visste nu så skulle jag garanterat inte ha använt
en virtuell maskin för projektet då det har lagt till ett extra lager av
komplexitet som skulle ha kunnat undvikas. Det hade varit mycket enklare att
använda en separat dator som helt kunde dedikeras till experiment och även
offras utan alltför stora förluster i fall av skadeverkan till orsakad av
crackers och annat otäckt som en direkt internetanslutning medför.

Med tanke på att hela hemnätverkets säkerhet står på spel tycker jag också att
bättre dokumentation och rådgivning borde ha funnits för konfiguration av
brandväggar.  Även om huvudpoängen är att ``lära sig att lära'' och en väldigt
stor del av arbete inom IT kretsar just kring att snabbt hitta rätt
information, borde några fler riktlinjer funnits tillgängliga.  I kompendiet
tipsades om \texttt{shorewall6}, men att konfigurera det på egen hand med den
experimentuppställning som använts under labben är inte en trivial övning.  Det
faktum att ``open source''-projekt generellt kan ha bristande dokumentation
\cite{baddoc:1} \cite{baddoc:2} \cite{baddoc:3} \cite{baddoc:4} t.ex. inte
uppdaterad eller motstridig, gör egna eftersökningar så mycket svårare.

Kanske antas väldigt goda tidigare kunskaper inom området, kanske skulle detta
belysas tydligare i instruktionerna och introduktionen.  För många kanske det
här projektet är både den första och sista gången de gör någon slags konfiguration av
brandväggar i Linux-miljö, och kanske riskerar de hela familjens enheter medan
de lär sig.  Det vore bra om man kunde öva i någon form av sandlåda, där misstag
inte har fullt lika stora konsekvenser.
Samtidigt ger det ``skarpa läget'' stark motivation till förbättring och
demonstrerar tydligt hur viktigt det verkligen är med god säkerhet..


Ett andra problem jag stötte på var praktisk testning, särskilt test av
\texttt{SSH}-åtkomst.  Jag försökte ansluta till skolans servrar för att
därifrån tunnla tillbaka, men då anslutningen inte gick över IPv6 hela vägen
(antar jag?) så lyckades jag inte.  Jag registerade även ett konto på
\texttt{sdf.org}, en gratis öppen \texttt{UNIX}-server, för att kunna testa,
men då de inte erbjöd användning av IPv6 vid \texttt{SSH}-anslutningar gick det
inte heller.


% ~~~~~~~~~~~~~~~~~~~~~~~~~~~~~~~~~~~~~~~~~~~~~~~~~~~~~~~~~~~~~~~~~~~~~~~~~~~~~~
\section{Slutsatser}
På det stora hela har jag greppat koncepten som presenterats, och har lärt mig
väldigt mycket om relaterade ämnesområden. Projektet har gett tillfälle att öva
på mycket viktiga och grundläggande koncept som knyter samman allt kursen har
tagit upp på ett sätt som demonstrerar innehållets relevans för olika områden.
