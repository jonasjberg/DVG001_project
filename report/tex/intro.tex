% ______________________________________________________________________________
%
% DVG001 -- Introduktion till Linux och små nätverk
%                                     Projektarbete
% ~~~~~~~~~~~~~~~~~~~~~~~~~~~~~~~~~~~~~~~~~~~~~~~~~
% Author:   Jonas Sjöberg
%           tel12jsg@student.hig.se
%
% Date:     2016-06-07 -- 2016-06-13
%
% License:  Creative Commons Attribution 4.0 International (CC BY 4.0)
%           <http://creativecommons.org/licenses/by/4.0/legalcode>
%           See LICENSE.md for additional licensing information.
% ______________________________________________________________________________


\section{Inledning}
% Skriv en kort inledning här som beskriver kortfattat vad rapporten handlar
% om. Den skall vara orienterande om Bakgrund och Syfte.
% TODO: ..


% ______________________________________________________________________________
\subsection{Bakgrund}
% Beskriv lite mer ingående om bakgrunden till uppgiften, vad den handlar om.
Laborationen bygger vidare på de föregående laborationerna och behandlar vidare
praktisk användning av \texttt{IPv6} genom installation och konfiguration av
en server för direkt åtkomst genom olika protokoll över internet.

Den virtuella maskin som skapades tidigare under kursens gång används under
laborationen och kommer bland annat att få agera server och demonstrera vanligt
förekommande verktyg och program för nätverksadministration.

% ______________________________________________________________________________
\subsection{Syfte}
% Skriv lite mer ingående om syftet med uppgiften.
Syftet med laborationen är att vidare demonstrera och ge ytterligare tillfälle
till att öva praktisering av systemadministration, särskilt relaterat till
servrar och nätverk.

% ______________________________________________________________________________
\subsection{Arbetsmetod}
% Hur kommer ni att arbeta?  Detta är en lite längre text än den rent
% orienterande texten i Planering och genomförande ovan.

Nedan följer en preliminär redogörelse för den experimentuppställning som
användes under laborationen:

\begin{itemize}
  \item Laborationen utförs på en \texttt{ProBook-6545b} laptop som kör
        \texttt{Xubuntu 16.04} på kerneln \texttt{Linux 4.4.0-21}.  Under
        tidigare laborationer körde värdsystemet ett 32-bitars operativsystem.
        Innan denna laboration uppgraderades värdsystemets operativsystem och
        då till en 64-bitars version. Förändringen i arkitektur har ännu inte
        krävt några justeringar av den virtuella maskinen som upprättats för
        kursarbetet.

  \item Rapporten skrivs i \LaTeX\  som kompileras till pdf med \texttt{latexmk}.
        Detta sker på värdsystemet.

  \item Virtualisering sker med \texttt{Oracle VirtualBox} version
        \texttt{5.0.18\_Ubuntu r106667}.

  \item Utveckling av programkod och testkörning sker i gästsystemet som kör
        \texttt{Debian 7.3 (jessie)} på kerneln \texttt{Linux 3.16.0-4}.

  \item Både rapporten och eventuell kod skrivs med texteditorn \texttt{Vim}.

  \item För versionshantering av både rapporten och programkod används \texttt{Git}.
    \begin{itemize}
      \item Källkod till programmet och rapporten finns att hämta på:

            \url{https://github.com/jonasjberg/DVG001\_project}

      \item Hämta hem repon genom att exekvera följande från kommandoraden:
            
            \texttt{git clone git@github.com:jonasjberg/DVG001\_project.git}

    \end{itemize}
\end{itemize}


