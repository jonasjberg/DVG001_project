% ______________________________________________________________________________
%
% DVG001 -- Introduktion till Linux och små nätverk
%                                     Projektarbete
% ~~~~~~~~~~~~~~~~~~~~~~~~~~~~~~~~~~~~~~~~~~~~~~~~~
% Author:   Jonas Sjöberg
%           tel12jsg@student.hig.se
%
% Date:     2016-06-07 -- 2016-06-13
%
% License:  Creative Commons Attribution 4.0 International (CC BY 4.0)
%           <http://creativecommons.org/licenses/by/4.0/legalcode>
%           See LICENSE.md for additional licensing information.
% ______________________________________________________________________________


% ______________________________________________________________________________
\section{Konfiguration av servern}
\subsection{Konfiguration av fjärråtkomst}
En del av projektets redovisning kräver att läraren ska kunna logga in på servern
med SSH genom att använda serverns IPv6-adress. Här förbereds inför detta.


% ______________________________________________________________________________
\subsection{Registrering av DNS-tjänst}
För att underlätta test och ge en mer lätthanterlig adress registrerades ett
konto hos \texttt{https://dynv6.com}, som erbjuder en gratis dynamisk
DNS-tjänst. Detta efter tips på kursens Blackboard-forum.

På så vis kan servern nås med en mer ``\emph{human-readable}'' adress,
som i det här fallet valdes att vara \texttt{http://jonasjberg.dynv6.net}.

För att testa adressen användes en IPv6-proxy på \texttt{http://www.ipv6proxy.net}.
Resultatet av testet visas i Figur~\ref{fig:05}.

\screenshot{include/05_ipv6proxy}
           {Skärmdump på test av anslutning till \texttt{dynv6.net}-adressen.}
           {Skärmdump på test av anslutning till \texttt{dynv6.net}-adressen 
					  \texttt{http://jonasjberg.dynv6.net} genom en IPv6-proxy.}
           {fig:05}


% ______________________________________________________________________________
\subsubsection{Ny användare \texttt{higjxn}}
En ny användare som läraren kan använda vid inloggning skapas med kommandot
\texttt{adduser higjxn}.

Den nya användaren läggs till i grupperna \texttt{adm} och \texttt{sudo} för
att möjliggöra kontroll av serverns konfiguration och loggar. Detta görs med
kommandot \texttt{adduser}.


% ______________________________________________________________________________
\subsection{Test av fjärråtkomst}
Test av inloggning över SSH med användaren \texttt{higjxn} visas i
Programlistning~\ref{listing:ssh-higjxn}.

\shellsource{include/cmd_ssh-higjxn}
            {Test av SSH med användaren \texttt{higjxn}.}
            {listing:ssh-higjxn}

Ett andra test av åtkomst över SSH med hjälp av ett onlineverktyg på adressen
\texttt{http://www.infobyip.com/sshservertest.php} visas i Figur~\ref{fig:06}.

\screenshot{include/06_ssh-test}
           {Skärmdump på test av SSH-server.}
					 {Skärmdump på test av SSH-anslutning med hjälp av onlineverktyget 
				    \texttt{http://www.infobyip.com/sshservertest.php}.}
           {fig:06}


% ______________________________________________________________________________
\subsection{Servern som lokal IPv6-router}
Debian-servern kan konfigureras för att agera router åt övriga datorer i det
lokala nätverket. På så vis kan de också göra IPv6-anslutningar.

Till att börja med ändrades filen \texttt{/etc/sysctl.conf}. Den ändrade
raden visas i Programlistning~\ref{listing:sysctl}.

\configsource{include/conf_sysctl}
             {Utdrag ur konfigurationsfilen \texttt{/etc/sysctl.conf}.}
             {listing:sysctl}
