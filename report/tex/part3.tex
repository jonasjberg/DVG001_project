% ______________________________________________________________________________
%
% DVG001 -- Introduktion till Linux och små nätverk
%                                     Projektarbete
% ~~~~~~~~~~~~~~~~~~~~~~~~~~~~~~~~~~~~~~~~~~~~~~~~~
% Author:   Jonas Sjöberg
%           tel12jsg@student.hig.se
%
% Date:     2016-06-07 -- 2016-06-13
%
% License:  Creative Commons Attribution 4.0 International (CC BY 4.0)
%           <http://creativecommons.org/licenses/by/4.0/legalcode>
%           See LICENSE.md for additional licensing information.
% ______________________________________________________________________________


% ______________________________________________________________________________
\section{Servern som lokal IPv6-router}
Debian-servern kan konfigureras för att agera router åt övriga datorer i det
lokala nätverket. På så vis kan de också göra IPv6-anslutningar.

Till att börja med ändrades filen \texttt{/etc/sysctl.conf}. Den ändrade
raden visas i Programlistning~\ref{listing:sysctl}.

\configsource{include/conf_sysctl}
             {Utdrag ur konfigurationsfilen \texttt{/etc/sysctl.conf}.}
             {listing:sysctl}

\subsection{Brandvägg}
När servern börjat routa trafik från IPv6-tunneln till det lokala
nätverket får varje enhet på det lokala nätverket en egen publik IPv6-adress.
De lämnas då i ett mycket sårbart läge för eventuella intrång och det
är viktigt att en brandvägg skyddar dem. 

Inställningar i brandväggen gjordes efter instruktioner från flera källor, 
bland annat \cite{ipv6:settingup}, \cite{debian:networkconfig}


% TODO: Skriv klart ..


